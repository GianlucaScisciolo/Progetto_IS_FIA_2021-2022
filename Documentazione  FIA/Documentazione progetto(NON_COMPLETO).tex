\documentclass[10pt,a4paper]{article}
\usepackage[utf8]{inputenc}
\usepackage{amsmath}
\usepackage{amsfonts}
\usepackage{amssymb}
\usepackage{lipsum,enumitem}
\usepackage{graphicx}
\newcommand{\subsubsubsection}[1]{\paragraph{#1}\mbox{}\\}
\setcounter{secnumdepth}{4}
\setcounter{tocdepth}{4}

\begin{document}
  
  % Contenuti
  \tableofcontents
  
  % Sezione 1
  \section{Definizione del progetto}
    \label{definizioneDelProgettoSection}
    
    % Sezione 1.1
    \subsection{L'idea del progetto di IS}
      \label{ideaDelProgettoDiISSubsection}
        L'idea del progetto di IS e' quello di permettere ad uno studente di trovare dei 
        percorsi formativi da apprendere. Questi percorsi formativi sono insegnati da dei 
        formatori, in particolare, ogni percorso formativo e' sostenuto da uno studente in 
        una o piu' fasce orarie in uno o piu' giorni della settimana, ed ogni percorso 
        formativo e' insegnato da un solo formatore quindi, avremo dei percorsi formativi 
        individuali.
        
    % Sezione 1.2
    \subsection{L'idea del modulo intelligente di FIA}
      \label{ideaDelModuloIntelligenteDiFIASubsection}
        l'idea del modulo intelligente di FIA e' il seguente:\\
        Fornire ad uno studente un piano formativo personalizzato sulla base dei seguenti 
        parametri:\\
        \begin{enumerate}
          \item Gli interessi dello studente.
          \item I giorni della settimana in cui lo studente e' libero.
        \end{enumerate}
        
     % Sezione 1.3
    \subsection{Analisi del problema:}
      \label{analisiDelProblemaSubsection}
        Analisi del problema:\\
        Ambiente:
        \begin{itemize}
          \item completamente osservabile: in ogni momento l'agente ha a disposizione tutti i percorsi formativi;
          \item stocastico: ad ogni azione dell'agente non sappiamo quale sara' il prossimo stato;
          \item episodico;
          \item statico: l'ambiente e' sempre lo stesso;
          \item discreto: le percezioni dell'agente sono limitate poiche' avra' un numero limitato di specifiche;
          \item singolo: abbiamo un unico agente.
        \end{itemize}
        
        Sensori:
        \begin{itemize}
          \item gli interessi dello studente;
          \item i giorni della settimana in cui lo studente e' libero;
        \end{itemize}
        
        Attuatori:
        \begin{itemize}
          \item un vettore di 7 percorsi formativi mostrati a video.
        \end{itemize}
\end{document}













