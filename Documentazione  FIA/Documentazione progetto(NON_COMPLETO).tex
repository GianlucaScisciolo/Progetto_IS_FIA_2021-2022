\documentclass[10pt,a4paper]{article}
\usepackage[utf8]{inputenc}
\usepackage{amsmath}
\usepackage{amsfonts}
\usepackage{amssymb}
\usepackage{lipsum,enumitem}
\usepackage{graphicx}
\newcommand{\subsubsubsection}[1]{\paragraph{#1}\mbox{}\\}
\setcounter{secnumdepth}{4}
\setcounter{tocdepth}{4}

\begin{document}
  
  % Contenuti
  \tableofcontents
  
  % Sezione 1
  \section{Definizione del progetto}
    \label{definizioneDelProgettoSection}
    
    % Sezione 1.1
    \subsection{L'idea del progetto di IS}
      \label{ideaDelProgettoDiISSubsection}
        L'idea del progetto di IS e' quello di permettere ad uno studente di trovare dei 
        percorsi formativi da apprendere. Questi percorsi formativi sono insegnati da dei 
        formatori, in particolare, ogni percorso formativo e' sostenuto da uno studente in 
        una o piu' fasce orarie in uno o piu' giorni della settimana, ed ogni percorso 
        formativo e' insegnato da un solo formatore quindi, avremo dei percorsi formativi 
        individuali.%
        
    % Sezione 1.2
    \subsection{L'idea del modulo intelligente di FIA}
      \label{ideaDelModuloIntelligenteDiFIASubsection}
        l'idea del modulo intelligente di FIA e' il seguente:\\
        Fornire ad uno studente un piano formativo personalizzato sulla base dei seguenti 
        parametri:\\
        \begin{enumerate}
          \item Gli interessi dello studente.
          \item lo storico dei percorsi formativi sostenuti dallo studente.
          \item I giorni della settimana in cui lo studente e' libero.
        \end{enumerate}%
        
    % Sezione 1.3
    \subsection{Analisi del problema:}
      \label{analisiDelProblemaSubsection}
        Analisi del problema:\\
        \begin{itemize}
        \item Ambiente:
        \begin{itemize}
          \item[$\circ$] completamente osservabile: in ogni momento l'agente ha a disposizione tutti i percorsi formativi;
          \item[$\circ$] stocastico: ad ogni azione dell'agente non sappiamo quale sara' il prossimo stato;
          \item[$\circ$] episodico;
          \item[$\circ$] statico: l'ambiente e' sempre lo stesso;
          \item[$\circ$] discreto: le percezioni dell'agente sono limitate poiche' avra' un numero limitato di specifiche;
          \item[$\circ$] singolo: abbiamo un unico agente.
        \end{itemize}
        
        \item Sensori:
        \begin{itemize}
          \item[$\circ$] gli interessi dello studente;
          \item[$\circ$] lo storico dei percorsi formativi sostenuti dallo studente.
          \item[$\circ$] i giorni della settimana in cui lo studente e' libero;
        \end{itemize}
        
        \item Attuatori:
        \begin{itemize}
          \item[$\circ$] un vettore di 7 percorsi formativi mostrati a video.
        \end{itemize}
      \end{itemize}%
        
    % Sezione 1.4
    \subsection{Analisi dell'algoritmo:}
      \label{analisiDellAlgoritmo}
        Utilizziamo un algoritmo di ricerca.\\
        Non mi e' sembrato opportuno utilizzare un grafo, poiche' non ho trovato dei collegamenti opportuni 
        tra i vari nodi.\\
        Escludiamo le seguenti ricerche:
        \begin{itemize}
          \item Ricerca non informata.
          \item Ricerca informata.
        \end{itemize}
        Non mi e' sembrato opportuno utilizzare un albero, poiche' non ho trovato una gerarchia tra i vari nodi e 
        inoltre abbiamo soltanto un unico agente.\\
        Escludiamo la seguente ricerca:
        \begin{itemize}
          \item Ricerca con avversari.
        \end{itemize}
        Utilizziamo un algoritmo genetico.\\
        Poiche' la ricerca deve essere il piu' ottimale possibile, questo 
        potrebbe comportare un tempo piu' lungo rispetto ad una normale ricerca per appunto avvantaggiare l'efficienza.\\
        Nasce quindi la necessita' di avvisare lo studente che potrebbe essere necessario aspettare qualche secondo in piu'.\\
        Quindi, nell'arco della progettazione dell'algoritmo genetico, Prenderemo in considerazione la 
        seguente classifica numerata dal piu' importante al meno importante:
        \begin{enumerate}
          \item Completezza e ottimalita'.
          \item Complessita' temporale.
          \item Complessita' spaziale.
        \end{enumerate}
        Ovviamente, l'utente non vorra' aspettare molto tempo quindi, all'algoritmo gli daremo del tempo 
        non superiore ai 10 secondi, mentre la visualizzazione del piano formativo avverra' dopo massimo 12 secondi. 
\end{document}













