\documentclass[10pt,a4paper]{article}
\usepackage[utf8]{inputenc}
\usepackage{amsmath}
\usepackage{amsfonts}
\usepackage{amssymb}
\usepackage{lipsum,enumitem}
\usepackage{graphicx}
\usepackage{program}
\newcommand{\subsubsubsection}[1]{\paragraph{#1}\mbox{}\\}
\setcounter{secnumdepth}{4}
\setcounter{tocdepth}{4}

\begin{document}
   
  % Contenuti
  \tableofcontents
  
  % Sezione 1
  \section{Definizione del progetto}
    \label{definizioneDelProgettoSection}
    
    % Sezione 1.1
    \subsection{L'idea del progetto di IS}
      \label{ideaDelProgettoDiISSubsection}
      L'idea del progetto di IS e' quello di permettere ad uno studente di trovare dei 
      percorsi formativi da apprendere. Questi percorsi formativi sono insegnati da dei 
      formatori, in particolare, ogni percorso formativo e' sostenuto da uno studente in 
      una o piu' fasce orarie in uno o piu' giorni della settimana, ed ogni percorso 
      formativo e' insegnato da un solo formatore quindi, avremo dei percorsi formativi 
      individuali.%
        
    % Sezione 1.2
    \subsection{L'idea del modulo intelligente di FIA}
      \label{ideaDelModuloIntelligenteDiFIASubsection}
      l'idea del modulo intelligente di FIA e' il seguente:\\
      Fornire ad uno studente un piano formativo personalizzato sulla base dei seguenti 
      parametri:\\
      \begin{enumerate}
        \item Gli interessi dello studente.
        \item lo storico dei percorsi formativi sostenuti dallo studente.
        \item I giorni della settimana in cui lo studente e' libero.
      \end{enumerate}%
        
    % Sezione 1.3
    \subsection{Analisi del problema}
      \label{analisiDelProblemaSubsection}
      Analisi del problema:\\
      \begin{itemize}
        \item Ambiente:
        \begin{itemize}
          \item[$\circ$] completamente osservabile: in ogni momento l'agente ha a disposizione tutti i percorsi formativi;
          \item[$\circ$] stocastico: ad ogni azione dell'agente non sappiamo quale sara' il prossimo stato;
          \item[$\circ$] episodico;
          \item[$\circ$] statico: l'ambiente e' sempre lo stesso;
          \item[$\circ$] discreto: le percezioni dell'agente sono limitate poiche' avra' un numero limitato di specifiche;
          \item[$\circ$] singolo: abbiamo un unico agente.
        \end{itemize}
        
        \item Sensori:
        \begin{itemize}
          \item[$\circ$] gli interessi dello studente;
          \item[$\circ$] lo storico dei percorsi formativi sostenuti dallo studente.
          \item[$\circ$] i giorni della settimana in cui lo studente e' libero;
        \end{itemize}
        
        \item Attuatori:
        \begin{itemize}
          \item[$\circ$] un vettore di massimo 7 percorsi formativi mostrati a video.
        \end{itemize}
      \end{itemize}%
        
    % Sezione 1.4
    \subsection{Analisi dell'algoritmo}
      \label{analisiDellAlgoritmoSubsection}
      Utilizziamo un algoritmo di ricerca.\\
      Non mi e' sembrato opportuno utilizzare un grafo, poiche' non ho trovato dei collegamenti opportuni 
      tra i vari nodi.\\
      Escludiamo le seguenti ricerche:
      \begin{itemize}
        \item Ricerca non informata.
        \item Ricerca informata.
      \end{itemize}
      Non mi e' sembrato opportuno utilizzare un albero, poiche' non ho trovato una gerarchia tra i vari nodi e 
      inoltre abbiamo soltanto un unico agente.\\
      Escludiamo la seguente ricerca:
      \begin{itemize}
        \item Ricerca con avversari.
      \end{itemize}
      Utilizziamo un algoritmo genetico.\\
      Poiche' la ricerca deve essere il piu' ottimale possibile, questo 
      potrebbe comportare un tempo piu' lungo rispetto ad una normale ricerca per appunto avvantaggiare l'efficienza.\\
      Nasce quindi la necessita' di avvisare lo studente che potrebbe essere necessario aspettare qualche secondo in piu'.\\
      Quindi, nell'arco della progettazione dell'algoritmo genetico, Prenderemo in considerazione la 
      seguente classifica numerata dal piu' importante al meno importante:
      \begin{enumerate}
        \item Completezza e ottimalita'.
        \item Complessita' temporale.
        \item Complessita' spaziale.
      \end{enumerate}
      Ovviamente, l'utente non vorra' aspettare molto tempo quindi, all'algoritmo gli daremo del tempo 
      non superiore ai 10 secondi, mentre la visualizzazione del piano formativo avverra' dopo massimo 12 secondi. 
        
  % Sezione 2
  \section{Progettazione dell'algoritmo genetico}
    \label{progettazioneDellAlgoritmoGeneticoSection}
    
    % Sezione 2.1
    \subsection{Spazio degli stati}
      \label{spazioDegliStatiSubsection}
      Lo spazio degli stati lo ricaviamo da una interrogazione al database ricavandoci i percorsi formativi
      insegnati nei giorni della settimana in cui lo studente e' libero.\\
      Abbiamo quindi un vettore (Vettore stati) di N percorsi formativi in cui abbiamo l'id presente nel database:\\
      \begin{figure}[h!]
        \centering
        \caption{Vettore stati}
        \includegraphics[scale=0.5]{spazioDegliStati.jpg}
        \label{vettoreStati}
      \end{figure}\\
      
    % Sezione 2.2
    \subsection{Codifica degli individui}
      \label{codificaDegliIndividuiSubsection}
      Come codifica utilizzeremo un vettore di massimo 7 interi positivi (individuo).\\
      2 Esempi di individui:\\
      \begin{figure}[h!]
        \centering
        \caption{Esempio individui}
        \includegraphics[scale=0.5]{esempiIndividui.jpg}
        \label{esempioIndividui}
      \end{figure}\\
      Ogni intero rappresenta una posizione di un percorso formativo all'interno del vettore stati.\\
      Perche' massimo 7? perche' vogliamo ritornare un sottoinsieme di 7 percorsi formativi (piano formativo)
      ordinati in base al giorno e all'orario.\\
      Questo sara' il DNA della soluzione composta da massimo 7 geni. Quindi, ogni individuo non e' 
      altro che un piano formativo.\\
      Ora, considerando il vettore stati e N il numero di percorsi formativi presente nel vettore stati:\\
      \begin{itemize}
        \item se $ N = 0 $ allora ritorno: non ho trovato nessun percorso formativo.
        \item se $ 0 < N \leq 7 $ allora ritorno tutti i percorsi formativi senza eseguire l'algoritmo.
        \item se $ N > 7 $ allora eseguo l'algoritmo genetico e ritorno l'individuo trovato.
      \end{itemize}
      
    % Sezione 2.3
    \subsection{Passi dell'algoritmo genetico}
      \label{passiDellAlgoritmoGeneticoSubsection}
      Questi saranno i passi dell'algoritmo genetico:\\
      \begin{figure}[h!]
        \centering
        \caption{Passi dell'algoritmo genetico}
        \includegraphics[scale=0.5]{passiAlgoritmoGenetico.jpg}
        \label{passiAlgoritmoGenetico}
      \end{figure}\\
      
    % Sezione 2.4
    \subsection{Inizializzo la popolazione}
      \label{inizializzoLaPopolazioneSubsection}
      Sia N il numero di percorsi formativi presenti nel vettore stati, la nostra\\
      popolazione sara' composta da un numero K di individui tale che:\\ 
      $ K = \bigl\lfloor {N \over 3} \bigl\rfloor $.\\
      All'inizio, tutti i K individui saranno generati in maniera random.\\
      Esempi:
      \begin{figure}
        \centering
        \caption{Esempi popolazioni}
        \includegraphics[scale=0.5]{esempiPopolazioni.jpg}
        \label{esempiPopolazioni}
      \end{figure}
      \\\\\\\\\\\\\\\\\\\\\\\\\\\\\\\\\\\\\\\\
    % Sezione 2.5
    \subsection{Valuto la popolazione}
      \label{valutoLaPopolazioneSubsection}
      
      Come possiamo valutare la popolazione?\\
      Abbiamo bisogno di una funzione obiettivo (funzioneObiettivo(codifica)).\\
      Dalla funzione obiettivo ci ricaviamo una funzione di valutazione (funzioneValutazione(codifica)) che sara':
      \begin{itemize}
        \item uguale alla funzione obiettivo nel caso in cui quest'ultima non richiede molte risorse, 
        sia accessibile, non e' complessa e non e' impossibile da realizzare.
        \item un approssimazione della funzione obiettivo nel caso in cui il punto precedente non e' 
        vero. 
      \end{itemize}
      Dopo di che, una volta applicata la funzione di valutazione, ad ogni individuo gli sara' dato un valore di 
      partenza che con la funzione di fitness (fit(codifica)) avremo' il grado di sopravvivenza di ciascun 
      individuo. La funzione di fitness dipende dall'algoritmo di selezione utilizzato.\\
      Funzione obiettivo (funzioneObiettivo(codifica)) considerazioni:\\
      poiche' possono esistere 2 o piu' geni con lo stesso valore allora il punteggio sara' dato solo ai 
      geni con valore diverso, quindi, ci servira' un modo per calcolare il punteggio solo ai nodi 
      diversi (utilizzeremo una funzione chiamata contains, java la fornisce).\\
      Le seguenti funzioni possono cambiare (non sono ancora quelle definitive):\\
      Funzione obiettivo:\\
      \begin{figure}[h!]
        \centering
        \caption{funzioneObiettivo(codifica) -- pseudocodice:}
        \includegraphics[scale=0.5]{funzioneObiettivo pseudocodice.jpg}
        \label{funzioneObiettivoPseudocodice}
      \end{figure}
      \\\\\\\\\\\\\\\\\\\\\\\\\\\\\\\\\\\\\\\\\\\\\\\\\\\\
      
      
      getPunteggio(gene):\\
      con la funzione getPunteggio(gene):\\
      \begin{enumerate}
       \item Mi ricavo il percorso formativo.
       \item Con il percorso formativo ricavato, mi ricavo:
         \begin{enumerate}
           \item Il valoreInteresse.
           \item Il valoreStorico.
           \item Il valoreCosto
         \end{enumerate}
       \item Utilizzero' una delle seguenti funzioni matematiche:
         \begin{itemize}
           \item Somma lineare di tutti i parametri:\\
                 SP = valoreInteresse + valoreStorico + valoreCosto.\\
                 Soluzione migliore: la soluzione che ha il punteggio:\\
                 SMP = (15 + 10 + 5) * 7 = 210.           
           \item Somma lineare di tutti i parametri moltiplicati per una costante c:\\
                 SP = (valoreInteresse * c) + (valoreStorico * c) + (valoreCosto * c).\\
                 Soluzione migliore: la soluzione che ha il punteggio:\\
                 SMP = [(15 * c) + (10 * c) + (5 * c)] * 7 = [(15 + 10 + 5) * c] * 7 = [30 * c] * 7 =
                 30 * 7 * c =  210 * c.
           \item Somma lineare di tutti i parametri elevati per una costante d:\\
                 $SP = (valoreInteresse ^ d) + (valoreStorico ^ d) + (valoreCosto ^ d).$\\
                 Soluzione migliore: la soluzione che ha il punteggio:\\
                 $SMP = [(15 ^ d) + (10 ^ d) + (5 ^ d)] * 7$
           \item Il quadrato della somma lineare di tutti i parametri:\\
                 $SP = (valoreInteresse + valoreStorico + valoreCosto)^ 2.$\\
                 Soluzione migliore: la soluzione che ha il punteggio:\\
                 $SMP = [(15 + 10 + 5) ^ 2] * 7 = 900 * 7 = 6300$
         \end{itemize}
       \end{enumerate}
       Li analizzeremo successivamente scegliendo la funzione che darà risultati migliori.
       \begin{figure}[h!]
        \centering
        \caption{getPunteggio(gene) -- pseudocodice:}
        \includegraphics[scale=0.5]{getPunteggio pseudocodice.jpg}
        \label{funzioneObiettivoPseudocodice}
      \end{figure}
      
    % Sezione 2.6
    \subsection{Mi fermo? -- Criteri di arresto}
    \label{MiFermo?--CriteriDiArrestoSubsection}
    Criteri di arresto utilizzati:
    \begin{enumerate}
      \item Test Ottimalità: se raggiungiamo l’ottimo allora ci fermiamo.
      \item Tempo di Esecuzione: dopo 10 secondi l’algoritmo termina.
    \end{enumerate}

\end{document}













