\documentclass[10pt,a4paper]{article}
\usepackage[utf8]{inputenc}
\usepackage{amsmath}
\usepackage{amsfonts}
\usepackage{amssymb}
\usepackage{lipsum,enumitem}
\usepackage{graphicx}
\usepackage{program}
\newcommand{\subsubsubsection}[1]{\paragraph{#1}\mbox{}\\}
\setcounter{secnumdepth}{4}
\setcounter{tocdepth}{4}

\begin{document}
   
  % Contenuti
  \tableofcontents
  
  % Sezione 1
  \section{Definizione del progetto}
    \label{definizioneDelProgettoSection}
    
    % Sezione 1.1
    \subsection{L'idea del progetto di IS}
      \label{ideaDelProgettoDiISSubsection}
      L'idea del progetto di IS e' quello di permettere ad uno studente di trovare dei 
      percorsi formativi da apprendere. Questi percorsi formativi sono insegnati da dei 
      formatori, in particolare, ogni percorso formativo e' sostenuto da uno studente in 
      una o piu' fasce orarie in uno o piu' giorni della settimana, ed ogni percorso 
      formativo e' insegnato da un solo formatore quindi, avremo dei percorsi formativi 
      individuali.%
        
    % Sezione 1.2
    \subsection{L'idea del modulo intelligente di FIA}
      \label{ideaDelModuloIntelligenteDiFIASubsection}
      l'idea del modulo intelligente di FIA e' il seguente:\\
      Fornire ad uno studente un piano formativo personalizzato sulla base dei seguenti 
      parametri:\\
      \begin{enumerate}
        \item Gli interessi dello studente.
        \item le iscrizioni dello studente
        \item I giorni della settimana in cui lo studente e' libero.
      \end{enumerate}%
      Inoltre, verranno visualizzati anche un insieme di massimo 6 percorsi formativi suggeriti.
        
    % Sezione 1.3
    \subsection{Analisi del problema}
      \label{analisiDelProblemaSubsection}
      Analisi del problema:\\
      \begin{itemize}
        \item Ambiente:
        \begin{itemize}
          \item[$\circ$] completamente osservabile: in ogni momento l'agente ha a disposizione tutti i percorsi formativi;
          \item[$\circ$] stocastico: ad ogni azione dell'agente non sappiamo quale sara' il prossimo stato;
          \item[$\circ$] episodico: la scelta di una azione non dipende dalle scelte fatte in "episodi" precedenti;
          \item[$\circ$] statico: l'ambiente e' sempre lo stesso;
          \item[$\circ$] discreto: le percezioni dell'agente sono limitate poiche' avra' un numero limitato di specifiche;
          \item[$\circ$] singolo: abbiamo un unico agente.
        \end{itemize}
        
        \item Sensori:
        \begin{itemize}
          \item[$\circ$] gli interessi dello studente;
          \item[$\circ$] le iscrizioni dello studente;
          \item[$\circ$] i giorni della settimana in cui lo studente e' libero;
        \end{itemize}
        
        \item Attuatori:
        \begin{itemize}
          \item[$\circ$] un vettore di massimo 10 percorsi formativi mostrati a video:
          \begin{itemize}
            \item I 4 percorsi formativi con punteggio più alto faranno parte del piano formativo;
            \item I percorsi formativi rimanenti (diversi tra loro e che non fanno parte 
            del piano formativo) faranno parte dei percorsi formativi suggeriti. 
          \end{itemize}
        \end{itemize}
      \end{itemize}%
        
    % Sezione 1.4
    \subsection{Analisi dell'algoritmo}
      \label{analisiDellAlgoritmoSubsection}
      Utilizziamo un algoritmo di ricerca.\\
      Non mi e' sembrato opportuno utilizzare un grafo, poiche' non ho trovato dei collegamenti opportuni 
      tra i vari nodi.\\
      Escludiamo le seguenti ricerche:
      \begin{itemize}
        \item Ricerca non informata.
        \item Ricerca informata.
      \end{itemize}
      Non mi e' sembrato opportuno utilizzare un albero, poiche' non ho trovato una gerarchia tra i vari nodi e 
      inoltre abbiamo soltanto un unico agente.\\
      Escludiamo la seguente ricerca:
      \begin{itemize}
        \item Ricerca con avversari.
      \end{itemize}
      Utilizziamo un algoritmo genetico.\\
      Poiche' la ricerca deve essere il piu' ottimale possibile, questo 
      potrebbe comportare un tempo piu' lungo rispetto ad una normale ricerca per appunto avvantaggiare l'efficienza.\\
      Nasce quindi la necessita' di avvisare lo studente che potrebbe essere necessario aspettare qualche secondo in piu'.\\
      Quindi, nell'arco della progettazione dell'algoritmo genetico, Prenderemo in considerazione la 
      seguente classifica numerata dal piu' importante al meno importante:
      \begin{enumerate}
        \item Completezza e ottimalita'.
        \item Complessita' temporale.
        \item Complessita' spaziale.
      \end{enumerate}
      Ovviamente, l'utente non vorra' aspettare molto tempo quindi, all'algoritmo gli daremo del tempo 
      non superiore ai 5 secondi + 2 secondi per la visualizzazione del piano formativo. Totale: 7 secondi.
        
  % Sezione 2
  \section{Progettazione dell'algoritmo genetico}
    \label{progettazioneDellAlgoritmoGeneticoSection}
    
    % Sezione 2.1
    \subsection{Spazio degli stati}
      \label{spazioDegliStatiSubsection}
      Lo spazio degli stati lo ricaviamo da una interrogazione al database: consideriamo solamente gli stati che hanno i percorsi formativi 
      che sono insegnati nei giorni liberi dello studente e non fanno parte delle sue iscrizioni.\\
      Abbiamo quindi un vettore (Vettore spazio stati) di N stati:\\
      \begin{figure}[h!]
        \centering
        \caption{Vettore spazio stati}
        \includegraphics[scale=0.5]{spazioDegliStati.jpg}
        \label{vettoreStati}
      \end{figure}
      
      \newpage
      
    % Sezione 2.2
    \subsection{Codifica degli individui}
      \label{codificaDegliIndividuiSubsection}
      Come codifica utilizzeremo un vettore di massimo 10 stati e, ogni stato della codifica rappresenta un gene.\\
      2 Esempi di individui:
      \begin{figure}[h!]
        \centering
        \caption{Esempio individui}
        \includegraphics[scale=0.5]{esempiIndividui.jpg}
        \label{esempioIndividui}
      \end{figure}\\
      Ogni stato (o gene) di un individuo è lo stato i-esimo presente nel vettore spazio stati.\\
      Perche' massimo 10? perche' vogliamo ritornare un sottoinsieme di 10 percorsi formativi di cui i 
      4 percorsi formativi con punteggio più alto ordinati in base al giorno e all'orario faranno parte 
      del piano formativo e i percorsi formativi rimanenti diversi tra loro e non presenti 
      nel piano formativo faranno parte dei consigliati.\\
      Questo sara' il DNA della soluzione composta da massimo 10 geni. Quindi, ogni individuo non e' 
      altro che un piano formativo con dei percorsi formativi consigliati.\\
      Ora, considerando il vettore spazio stati e N il numero di percorsi formativi presente 
      nel vettore spazio stati:\\
      \begin{itemize}
        \item se $ N = 0 $ allora ritorno un insieme vuoto senza eseguire l'algoritmo.
        \item se $ 0 < N \leq 10 $ allora ritorno tutti i percorsi formativi senza eseguire 
        l'algoritmo.
        \item se $ N > 10 $ allora eseguo l'algoritmo genetico e ritorno l'individuo trovato.
      \end{itemize}
      N.B: Ogni stato contiene solamente un percorso formativo, quindi, alla fine della computazione dell'algoritmo, 
      estrarro' i vari percorsi formativi presenti negli stati (o geni) dell'individuo migliore trovato.
      
      \newpage
      
    % Sezione 2.3
    \subsection{Passi dell'algoritmo genetico}
      \label{passiDellAlgoritmoGeneticoSubsection}
      Questi saranno i passi dell'algoritmo genetico:
      \begin{figure}[h!]
        \centering
        \caption{Passi dell'algoritmo genetico}
        \includegraphics[scale=0.5]{passiAlgoritmoGenetico.jpg}
        \label{passiAlgoritmoGenetico}
      \end{figure}\\
            
    % Sezione 2.4
    \subsection{Inizializzo la popolazione}
      \label{inizializzoLaPopolazioneSubsection}
      Sia N il numero di percorsi formativi presenti nel vettore spazio stati, la nostra\\
      popolazione sara' composta da un numero K di individui tale che:\\ 
      $ K = \bigl\lfloor {N \over 3} \bigl\rfloor $.\\
      All'inizio, tutti i K individui saranno generati in maniera random (tranne nel caso in cui N $\leq$ 10).\\
      Esempi:
      \begin{figure}
        \centering
        \caption{Esempi popolazioni}
        \includegraphics[scale=0.5]{esempiPopolazioni.jpg}
        \label{esempiPopolazioni}
      \end{figure}
      
      \newpage
      
    % Sezione 2.5
    \subsection{Valuto la popolazione (Da vedere)}
      \label{valutoLaPopolazioneSubsection}
      
      Come possiamo valutare la popolazione?\\
      Abbiamo bisogno di una funzione obiettivo (funzioneObiettivo(codifica)).\\
      Dalla funzione obiettivo ci ricaviamo una funzione di valutazione (funzioneValutazione(codifica)) che sara':
      \begin{itemize}
        \item uguale alla funzione obiettivo nel caso in cui quest'ultima non richiede molte risorse, 
        sia accessibile, non e' complessa e non e' impossibile da realizzare.
        \item un approssimazione della funzione obiettivo nel caso in cui il punto precedente non e' 
        vero. 
      \end{itemize}
      Dopo di che, una volta applicata la funzione di valutazione, ad ogni individuo gli sara' dato un valore di 
      partenza che con la funzione di fitness (fitness(codifica)) avremo' il grado di sopravvivenza di ciascun 
      individuo. La funzione di fitness dipende dall'algoritmo di selezione utilizzato.\\
      Funzione obiettivo (funzioneObiettivo(codifica)) considerazioni:\\
      poiche' possono esistere 2 o piu' geni con lo stesso valore allora il punteggio sara' dato solo ai 
      geni con valore diverso, quindi, ci servira' un modo per calcolare il punteggio solo ai nodi 
      diversi (utilizzeremo una funzione chiamata contains, java la fornisce).\\
      Le seguenti funzioni possono cambiare (non sono ancora quelle definitive):\\
      Funzione obiettivo:\\
      \begin{figure}[h!]
        \centering
        \caption{funzioneObiettivo(codifica) -- pseudocodice:}
        \includegraphics[scale=0.5]{funzioneObiettivo pseudocodice.jpg}
        \label{funzioneObiettivoPseudocodice}
      \end{figure}
      \\\\\\\\\\\\\\\\\\\\\\\\\\\\\\\\\\\\\\\\\\\\\\\\\\\\
      
      
      getPunteggio(gene):\\
      con la funzione getPunteggio(gene):\\
      \begin{enumerate}
       \item Mi ricavo il percorso formativo.
       \item Con il percorso formativo ricavato, mi ricavo:
         \begin{enumerate}
           \item Il valoreInteresse.
           \item Il valoreStorico.
           \item Il valoreCosto
         \end{enumerate}
       \item Utilizzero' una delle seguenti funzioni matematiche:
         \begin{itemize}
           \item Somma lineare di tutti i parametri:\\
                 SP = valoreInteresse + valoreStorico + valoreCosto.\\
                 Soluzione migliore: la soluzione che ha il punteggio:\\
                 SMP = (15 + 10 + 5) * 7 = 210.           
           \item Somma lineare di tutti i parametri moltiplicati per una costante c:\\
                 SP = (valoreInteresse * c) + (valoreStorico * c) + (valoreCosto * c).\\
                 Soluzione migliore: la soluzione che ha il punteggio:\\
                 SMP = [(15 * c) + (10 * c) + (5 * c)] * 7 = [(15 + 10 + 5) * c] * 7 = [30 * c] * 7 =
                 30 * 7 * c =  210 * c.
           \item Somma lineare di tutti i parametri elevati per una costante d:\\
                 $SP = (valoreInteresse ^ d) + (valoreStorico ^ d) + (valoreCosto ^ d).$\\
                 Soluzione migliore: la soluzione che ha il punteggio:\\
                 $SMP = [(15 ^ d) + (10 ^ d) + (5 ^ d)] * 7$
           \item Il quadrato della somma lineare di tutti i parametri:\\
                 $SP = (valoreInteresse + valoreStorico + valoreCosto)^ 2.$\\
                 Soluzione migliore: la soluzione che ha il punteggio:\\
                 $SMP = [(15 + 10 + 5) ^ 2] * 7 = 900 * 7 = 6300$
         \end{itemize}
       \end{enumerate}
       Li analizzeremo successivamente scegliendo la funzione che darà risultati migliori.
       \begin{figure}[h!]
        \centering
        \caption{getPunteggio(gene) -- pseudocodice:}
        \includegraphics[scale=0.5]{getPunteggio pseudocodice.jpg}
        \label{getPunteggioPseudocodice}
      \end{figure}
      
      \newpage
      
    % Sezione 2.6
    \subsection{Mi fermo? -- Criteri di arresto}
    \label{MiFermo?--CriteriDiArrestoSubsection}
    Criteri di arresto utilizzati:
    \begin{enumerate}
      \item Test Ottimalità: se raggiungiamo l’ottimo allora ci fermiamo.
      \item Tempo di Esecuzione: dopo 5 secondi l’algoritmo termina.
    \end{enumerate}
    
    % Sezione 2.7
    \subsection{Selezione}
    \label{Selezione}
    Esistono vari tipi di selezione. Andiamo ad analizzarli:
    - Roulette Wheel Selection:
    \begin{figure}[h!]
      \centering
      \caption{Roulette Wheel Selection}
      \includegraphics[scale=0.5]{Roulette Wheel Selection.png}
      \label{RouletteWheelSelection}
    \end{figure}
    \begin{enumerate}
      \item fitness$_{i}(codifica) =\frac{f_{i}(codifica)}{\sum\limits_{j = 1}^{K} {f_{j} (codifica)}}$.
      \item andiamo a girare la ruota un numero di volte uguale al numero di individui presenti nella 
            popolazione.
    \end{enumerate} 
    
    \newpage
    
    Rank Selection:\\\\
    $ fitness(codifica) = \frac {(\mid P \mid - rank(codifica) + 1)} {\sum{rank(codifica)}}$\\
    E’ applicabile anche con valutazioni negative.\\
    Costa più della Roulette Wheel.\\
    
    K-way Tournament Selection:\\
    Fissato un valore $1 < K < \mid P \mid$ arbitrario, si selezionano casualmente K individui per formare un 
    cosiddetto "torneo", all’interno del quale il migliore (con valutazione massima) passerà la 
    selezione. Si ripete per $M < \mid P \mid$ tornei, quindi fino ad ottenere un totale di M individui 
    selezionati.\\\\
    
    Quale tipo di selezione utilizzeremo?\\
    Roulette Wheel Selection: poco ideale per popolazioni piccole.\\
    Rank selection: troppo lenta.\\
    Scegliamo la k-way Tournament Selection.\\
    
    % Sezione 2.8
    \subsection{Crossover}
    \label{Crossover}
    Single-point Crossover:\\
    Punto di taglio  = numero casuale tra 1 e 9.\\
    Esempio: 5
    \begin{figure}[h!]
      \centering
      \caption{Single-point Crossover}
      \includegraphics[scale=0.5]{Single-point Crossover.png}
      \label{Single-pointCrossover}
    \end{figure}
    
    
    Two-point Crossover:\\
    Punti di taglio: 2 numeri casuali tra 1 e 9 diversi tra di loro.\\
    Esempio: 2, 6
    \begin{figure}[h!]
      \centering
      \caption{Two-point Crossover}
      \includegraphics[scale=0.5]{Two-point Crossover.png}
      \label{Two-pointCrossover}
    \end{figure}
    
    \newpage
    
    K-point Crossover:\\
    Punti di taglio: K numeri casuali tra 1 e 9 diversi tra loro. 1 $\leq$ K $\leq$ 9.\\
    Esempio: K = 4 $\rightarrow$ 4 numeri casuali tra 1 e 9 diversi tra di loro.\\
    Esempio: 1, 3, 5, 6
    \begin{figure}[h!]
      \centering
      \caption{K-point Crossover}
      \includegraphics[scale=0.5]{K-point Crossover.png}
      \label{K-pointCrossover}
    \end{figure}
    
    Uniform Crossover:\\
    Non si compie alcun taglio, ma il gene i-esimo di entrambi i figli sarà pari al gene i-esimo di uno 
    dei due genitori. Dato che la scelta tra i due genitori è uniforme, si avranno figli 
    con circa 50\% del materiale del primo genitore e 50\% del secondo genitore.\\
    Esempio:
    \begin{figure}[h!]
      \centering
      \caption{Uniform Crossover}
      \includegraphics[scale=0.5]{Uniform Crossover.png}
      \label{UniformCrossover}
    \end{figure}
    
    Quale tipo di Crossover utilizzeremo?\\
    Il Single-point Crossover con codifiche piccole non è l’ideale.\\
    Il Two-point Crossover crea più diversità ma abbiamo sempre una codifica piccola.\\
    Il K-point Crossover: Quale deve essere il valore ideale di K?\\
    Per levarci questo problema, scegliamo l’Uniform Crossover. In modo che non facciamo nessun taglio 
    e c’è più diversità.\\
    
    % Sezione 2.9
    \subsection{Mutazione}
    \label{Mutazione}
    Random resetting:\\
    modifica casuale di un gene in un altro gene.\\
    Esempio:\\
    N = 30\\
    \begin{enumerate}
      \item Scegliamo un gene casuale da mutare: il gene in posizione 2: S$_{14}$.
      \item Scegliamo uno stato casuale dal vettore spazio stati diverso dallo stato S$_{14}$: S$_{5}$.
      \item Mutiamo il gene: S$_{14} \rightarrow$ S$_{5}$.
    \end{enumerate}
    \begin{figure}[h!]
      \centering
      \caption{Random Resetting}
      \includegraphics[scale=0.5]{Random Resetting.png}
      \label{Random Resetting}
    \end{figure}
    
    Swap:\\
    scambiamo di posto 2 geni scelti casualmente.\\
    Esempio:\\
    \begin{enumerate}
    	\item scegliamo casualmente 2 geni in posizione i e j: 
    	\begin{enumerate}
    		\item i = 1;
    		\item j = 5.
    	\end{enumerate}
    	\item li scambiamo di posto.
    	\begin{enumerate}
    		\item i = 5;
    		\item j = 1.
    	\end{enumerate}
    \end{enumerate}
    
    \begin{figure}[h!]
      \centering
      \caption{Swap}
      \includegraphics[scale=0.5]{Swap.png}
      \label{Swap}
    \end{figure}
    
    Scramble:\\
    selezioniamo una sottosequenza casuale di geni e la permutiamo casualmente.
    Esempio:\\
    \begin{enumerate}
      \item Scegliamo casualmente la sottosequenza 2 → 3 → 4 → 5.
      \item La permutiamo casualmente: 4 → 5 → 2 → 3.
    \end{enumerate}
    \begin{figure}[h!]
      \centering
      \caption{Scramble}
      \includegraphics[scale=0.5]{Scramble.png}
      \label{Scramble}
    \end{figure}
     
    Inversion:\\
    selezioniamo una sottosequenza casuale di geni e la invertiamo.\\
    Esempio:\\
    \begin{enumerate}
      \item Scegliamo casualmente la sottosequenza 2 → 3 → 4 → 5.
      \item La invertiamo: 5 → 4 → 3 → 2.
    \end{enumerate}
    \begin{figure}[h!]
      \centering
      \caption{Inversion}
      \includegraphics[scale=0.5]{Inversion.png}
      \label{Inversion}
    \end{figure}
    
    Quale tipo di mutazione utilizzeremo?\\
    Ricordiamo che ogni gene rappresenta uno stato che ha come parametro un percorso formativo, quindi, non avrebbe senso utilizzare i 
    seguenti tipi di mutazione: Swap, Scramble e Inversion poiche' gli individui mutati saranno gli 
    stessi individui di prima solo che avranno gli stessi geni cambiati di posto.\\
    Quindi, prendiamo in considerazione la seguente mutazione: Random resetting poiche' un gene sarà 
    mutato in un altro gene diverso.
    
    % Sezione 2.10
    \subsection{Elitismo}
    \label{elitismoSubsection}
    Poiché gli individui d'elite non dovrebbero superare il 10\% della popolazione totale 
    allora l'elitismo lo utilizzeremo solo se il numero della popolazione totale è maggiore o uguale a 10, questo 
    perché il 10\% di 10 è uguale ad 1.
    Gli studenti di elite non subiranno ne la selezione, ne il crossover e ne la mutazione 
    e saranno sempre massimo 2 individui.
    
    % Sezione 2.11
    \subsection{Preserviamo l'ammissibilita'}
    \label{PreserviamoAmmissibilitaSubsection}
    Con il crossover e la mutazione si potrebbe correre il rischio che nelle nuove generazioni 
    potrebbero esistere degli individui non ammissibili. Nel nostro caso questo non puo' accadere perche':
    \begin{enumerate}
      \item Nel crossover abbiamo 2 genitori che formano 2 figli contenenti i geni 
      al 50\% del genitore 1 e al 50\% del genitore 2, poiche' ogni gene non è altro che uno stato che ha come attributo un percorso formativo esistente 
      allora non avremo mai dei percorsi formativi inesistenti.
      \item Nella mutazione: mutiamo un gene con un altro stato, questo nuovo stato sarà ancora un 
      stato valido poiche' e' presente nel vettore spazio stati e quindi di conseguenza abbiamo ancora un percorso formativo valido.
    \end{enumerate}
    
    % Sezione 2.12
    \subsection{Aggiorniamo la popolazione}
    \label{aggiorniamoPopolazioneSubsection}
    La nuova popolazione sara' costituita da N individui, ovvero:
    \begin{itemize}
      \item Sia A l'insieme degli individui di elite.
      \item Sia B l'insieme degli individui che hanno partecipato alla selezione, al crossover e alla 
      mutazione.
      \item Nuova popolazione = A U B.
    \end{itemize}
    Quindi aggiorniamo la vecchia popolazione con la nuova popolazione salvandoci pero' l'individuo 
    migliore della vecchia generazione se e solo se quest'ultimo e' migliore dell'ultimo individuo 
    migliore.\\
    N.B.: l'individuo migliore non farà parte della nuova generazione.
    
    % Sezione 2.13
    \subsection{Ottengo il miglior individuo}
    \label{ottengoMigliorIndividuoSubsection}
    Quando usciamo dal loop ritorneremo il miglior individuo, ovvero, uno dei seguenti individui:
    \begin{enumerate}
      \item Nel caso in cui raggiungo l'ottimo: prendiamo l'individuo migliore dell'ultima generazione ottenuta che verrà poi confrontato 
      con l'individuo che ci siamo salvati ad ogni generazione. Ritorniamo il migliore tra i 2;
      \item Nel caso in cui non raggiungo l'ottimo (il tempo scade):  ritorniamo l'individuo migliore trovato dalla 
      prima all'ultima generazione ottenuta (quello che ci siamo salvati ad ogni generazione).
    \end{enumerate}
    
    % Sezione 2.14
    \subsection{Riformulazione dei passi dell'algoritmo genetico}
    \label{riformulazioneDeiPassiAlgoritmoGeneticoSubsection}
    Alla luce della progettazione fatta, possiamo riformulare il tutto con i seguenti passi 
    dell'algoritmo genetico: 
    \begin{figure}[h!]
      \centering
      \caption{Riformulazione dei passi dell'algoritmo genetico}
      \includegraphics[scale=0.5]{Riformulazione passi algoritmo genetico.jpg}
      \label{Riformulazione passi algoritmo genetico}
    \end{figure}
      

\end{document}













