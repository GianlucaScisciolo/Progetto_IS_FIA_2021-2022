\documentclass[10pt,a4paper]{article}
\usepackage[utf8]{inputenc}
\usepackage{amsmath}
\usepackage{amsfonts}
\usepackage{amssymb}
\usepackage{lipsum,enumitem}
\usepackage{graphicx}
\newcommand{\subsubsubsection}[1]{\paragraph{#1}\mbox{}\\}
\setcounter{secnumdepth}{4}
\setcounter{tocdepth}{4}

\begin{document}
  
  % Contenuti
  \tableofcontents
  
  \section{Definizione del progetto}
    \label{definizioneDelProgettoSection}
    
    % Sezione 1.1
    \subsection{L'idea del progetto di IS}
      \label{ideaDelProgettoDiISSubsection}
        L'idea del progetto di IS � quello di permettere ad uno studente di trovare dei 
        percorsi formativi da apprendere. Questi percorsi formativi sono insegnati da dei 
        formatori, in particolare, ogni percorso formativo � sostenuto da uno studente in 
        una o pi� fasce orarie in uno o pi� giorni della settimana, ed ogni percorso 
        formativo � insegnato da un solo formatore quindi, avremo dei percorsi formativi 
        individuali.
        
    % Sezione 1.2
    \subsection{L'idea del modulo intelligente di FIA}
      \label{ideaDelModuloIntelligenteDiFIASubsection}
        l'idea del modulo intelligente di FIA � il seguente:\\
        Fornire ad uno studente un piano formativo personalizzato sulla base dei seguenti 
        parametri:\\
        \begin{enumerate}
          \item Gli interessi dello studente.
          \item I giorni della settimana in cui lo studente � libero.
        \end{enumerate}
  
\end{document}