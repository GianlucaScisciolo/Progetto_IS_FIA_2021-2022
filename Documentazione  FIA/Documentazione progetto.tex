\documentclass[10pt,a4paper]{article}
\usepackage[utf8]{inputenc}
\usepackage{amsmath}
\usepackage{amsfonts}
\usepackage{amssymb}
\usepackage{lipsum,enumitem}
\usepackage{graphicx}
\usepackage{program}
\newcommand{\subsubsubsection}[1]{\paragraph{#1}\mbox{}\\}
\setcounter{secnumdepth}{4}
\setcounter{tocdepth}{4}
\usepackage{color}
\usepackage{url}

\begin{document}
   
  % Contenuti
  \tableofcontents
  
   
   \subsection{Partecipanti}
   
   Partecipante: Gianluca Scisciolo.\\
   Matricola: 0512107432.\\
   Email: g.scisciolo@studenti.unisa.it
  
   \subsection{Collegamenti github}
   Collegamenti repository github:\\
   Repository modulo FIA:\\ \url{https://github.com/GianlucaScisciolo/formAct_FIA}\\
   Repositori progetto intero (contiene anche modulo FIA):\\ \url{https://github.com/fabio3649/formAct}\\
  
  \section{Definizione del progetto}
    \label{definizioneDelProgettoSection}
    
   
    \subsection{L'idea del progetto di IS}
      \label{ideaDelProgettoDiISSubsection}
      L'idea del progetto di IS e' quello di permettere ad uno studente di trovare dei 
      percorsi formativi da apprendere. Questi percorsi formativi sono insegnati da dei 
      formatori, in particolare, ogni percorso formativo e' sostenuto da uno studente in 
      una o piu' fasce orarie in uno o piu' giorni della settimana, ed ogni percorso 
      formativo e' insegnato da un solo formatore quindi, avremo dei percorsi formativi 
      individuali.%
        
   
    \subsection{L'idea del modulo intelligente di FIA}
      \label{ideaDelModuloIntelligenteDiFIASubsection}
      l'idea del modulo intelligente di FIA e' il seguente:\\
      Fornire ad uno studente un piano formativo personalizzato sulla base dei seguenti 
      parametri:\\
      \begin{enumerate}
        \item Gli interessi dello studente.
        \item le iscrizioni dello studente
        \item I giorni della settimana in cui lo studente e' libero.
      \end{enumerate}%
      Inoltre, verranno visualizzati anche un insieme di massimo 6 percorsi formativi suggeriti.
        
   
    \subsection{Analisi del problema}
      \label{analisiDelProblemaSubsection}
      Analisi del problema:\\
      \begin{itemize}
        \item Ambiente:
        \begin{itemize}
          \item[$\circ$] completamente osservabile: in ogni momento l'agente ha a disposizione tutti i percorsi formativi;
          \item[$\circ$] stocastico: ad ogni azione dell'agente non sappiamo quale sara' il prossimo stato;
          \item[$\circ$] episodico: la scelta di una azione non dipende dalle scelte fatte in "episodi" precedenti;
          \item[$\circ$] statico: l'ambiente e' sempre lo stesso;
          \item[$\circ$] discreto: le percezioni dell'agente sono limitate poiche' avra' un numero limitato di specifiche;
          \item[$\circ$] singolo: abbiamo un unico agente.
        \end{itemize}
        
        \item Sensori:
        \begin{itemize}
          \item[$\circ$] gli interessi dello studente;
          \item[$\circ$] le iscrizioni dello studente;
          \item[$\circ$] i giorni della settimana in cui lo studente e' libero;
        \end{itemize}
        
        \item Attuatori:
        \begin{itemize}
          \item[$\circ$] un vettore di massimo 10 percorsi formativi mostrati a video:
          \begin{itemize}
            \item I 4 percorsi formativi con punteggio più alto (diversi tra loro) faranno parte del piano formativo;
            \item I percorsi formativi rimanenti (diversi tra loro e che non fanno parte 
            del piano formativo) faranno parte dei percorsi formativi suggeriti. 
          \end{itemize}
        \end{itemize}
      \end{itemize}%
        
   
    \subsection{Analisi dell'algoritmo}
      \label{analisiDellAlgoritmoSubsection}
      Utilizziamo un algoritmo di ricerca.\\
      Non mi e' sembrato opportuno utilizzare un grafo, poiche' non ho trovato dei collegamenti opportuni 
      tra i vari nodi.\\
      Escludiamo le seguenti ricerche:
      \begin{itemize}
        \item Ricerca non informata.
        \item Ricerca informata.
      \end{itemize}
      Non mi e' sembrato opportuno utilizzare un albero, poiche' non ho trovato una gerarchia tra i vari nodi e 
      inoltre abbiamo soltanto un unico agente.\\
      Escludiamo la seguente ricerca:
      \begin{itemize}
        \item Ricerca con avversari.
      \end{itemize}
      Utilizziamo un algoritmo genetico.\\
      Poiche' la ricerca deve essere il piu' ottimale possibile, questo 
      potrebbe comportare un tempo piu' lungo rispetto ad una normale ricerca per appunto avvantaggiare l'efficienza.\\
      Nasce quindi la necessita' di avvisare lo studente che potrebbe essere necessario aspettare qualche secondo in piu'.\\
      Quindi, nell'arco della progettazione dell'algoritmo genetico, Prenderemo in considerazione la 
      seguente classifica numerata dal piu' importante al meno importante:
      \begin{enumerate}
        \item Completezza e ottimalita'.
        \item Complessita' temporale.
        \item Complessita' spaziale.
      \end{enumerate}
      Ovviamente, l'utente non vorra' aspettare molto tempo quindi, all'algoritmo gli daremo del tempo 
      non superiore a 4 secondi + il tempo necessario per l'ordinamento e la visualizzazione del piano formativo. Totale: circa 5 secondi.
        
    \section{Progettazione dell'algoritmo genetico}
    \label{progettazioneDellAlgoritmoGeneticoSection}
    
  
    \subsection{Spazio degli stati}
      \label{spazioDegliStatiSubsection}
      Lo spazio degli stati lo ricaviamo da una interrogazione al database: consideriamo solamente gli stati che hanno i percorsi formativi 
      che sono insegnati nei giorni liberi dello studente e che non fanno parte delle sue iscrizioni.\\
      Abbiamo quindi un vettore (Vettore spazio stati) di N stati:\\
      \begin{figure}[h!]
        \centering
        \caption{Vettore spazio stati}
        \includegraphics[scale=0.5]{spazioDegliStati.jpg}
        \label{vettoreStati}
      \end{figure}
      
      \newpage
      
  
    \subsection{Codifica degli individui}
      \label{codificaDegliIndividuiSubsection}
      Come codifica utilizzeremo un vettore di massimo 10 interi e, ogni intero della codifica rappresenta un gene dell'individuo.\\
      2 Esempi di individui:
      \begin{figure}[h!]
        \centering
        \caption{Esempio individui}
        \includegraphics[scale=0.5]{esempiIndividui.jpg}
        \label{esempioIndividui}
      \end{figure}\\
      Ogni gene di un individuo è un indice della posizione di uno stato nel vettore spazio stati.\\
      Quindi, ogni individuo non è altro che un vettore di massimo 10 interi che rappresentano gli indici delle posizioni degli stati 
      nel vettore spazio stati e, una volta finita la computazione, trasformiamo l'individuo in una soluzione.\\
      Soluzione: vettore di massimo 10 stati, ovvero, un piano formativo con dei percorsi formativi consigliati.\\ 
      Perche' massimo 10? perche' vogliamo ritornare un sottoinsieme di 10 percorsi formativi di cui i  primi 4 percorsi formativi diversi tra loro e con punteggio più alto, 
      ordinati in base al giorno e all'orario, faranno parte del piano formativo e i percorsi formativi rimanenti diversi tra loro e non presenti 
      nel piano formativo faranno parte dei consigliati.\\
      Esempio soluzioni:
      Rappresentazione dei 2 individui dell'esempio precedente sotto forma di soluzione:\\
      \begin{figure}[h!]
        \centering
        \caption{Esempio soluzioni}
        \includegraphics[scale=0.5]{esempiSoluzioni.jpg}
        \label{esempiSoluzioni}
      \end{figure}\\
      \newpage
     Ora, considerando il vettore spazio stati ed N il numero di percorsi formativi presenti nel vettore spazio stati:\\
      \begin{itemize}
        \item se $ N = 0 $ allora ritornero' una soluzione vuota senza eseguire l'algoritmo;
        \item se $ 0 < N \leq 10 $ allora ritornero' una soluzione vuota con tutti i percorsi formativi presenti nel vettore spazio stati 
        senza eseguire l'algoritmo;
        \item se $ N > 10 $ allora eseguiro' l'algoritmo genetico e ritornero' la soluzione migliore trovata data dall'individuo migliore trovato.
      \end{itemize}
      N.B: Ogni stato contiene solamente un percorso formativo, quindi, alla fine della computazione dell'algoritmo, estrarro' 
      i vari percorsi formativi presenti negli stati della soluzione migliore trovata.
     
  
     \subsection{Classe PercorsoFormativoEntity}
       \label{classePercorsoFormativoEntitySubsection}
       Ora, vediamo più nel dettaglio come è fatta la classe PercorsoFormativoEntity:\\
       \begin{figure}[h!]
         \centering
         \caption{Classe PercorsoFormativoEntity}
         \includegraphics[scale=0.5]{classePercorsoFormativoEntity.jpg}
         \label{classePercorsoFormativoEntity}
       \end{figure}\\
      
  
     \subsection{Classe Stato}
       \label{classeStatoSubsection}
       Ora, vediamo più nel dettaglio come è fatta la classe Stato:\\
       \begin{figure}[h!]
         \centering
         \caption{Classe Stato}
         \includegraphics[scale=0.5]{classeStato.jpg}
         \label{classeStato}
       \end{figure}\\
      
      \newpage
      
  
    \subsection{Passi dell'algoritmo genetico}
      \label{passiDellAlgoritmoGeneticoSubsection}
      Questi saranno i passi dell'algoritmo genetico:
      \begin{figure}[h!]
        \centering
        \caption{Passi dell'algoritmo genetico}
        \includegraphics[scale=0.5]{passiAlgoritmoGenetico.jpg}
        \label{passiAlgoritmoGenetico}
      \end{figure}\\
            
  
    \subsection{Inizializzo la popolazione}
      \label{inizializzoLaPopolazioneSubsection}
      Sia N = $\lvert$vettore spazio stati$\rvert$, la nostra popolazione sara' composta da un numero M di individui tale che:\\ 
      $ M = \bigl\lfloor {N \over 3} \bigl\rfloor $.\\
      Se $\bigl\lfloor {N \over 3} \bigl\rfloor $ è dispari allora:\\ 
      $ M = \bigl\lfloor {N \over 3} \bigl\rfloor $ + 1.\\
      All'inizio, tutti gli M individui saranno generati in maniera random (tranne nel caso in cui N $\leq$ 10).\\
      Esempi:
      \begin{figure}
        \centering
        \caption{Esempi popolazioni}
        \includegraphics[scale=0.5]{esempiPopolazioni.jpg}
        \label{esempiPopolazioni}
      \end{figure}
      
      \newpage
      
  
    \subsection{Valuto la popolazione}
      \label{valutoLaPopolazioneSubsection}
      
      Come possiamo valutare la popolazione?\\
      Abbiamo bisogno di una funzione di valutazione (evaluate(individuo)).\\
      Una volta applicata la funzione di valutazione, ad ogni individuo gli sara' dato un valore di 
      partenza che con la funzione di fitness (fitness(codifica)) avremo il grado di sopravvivenza di ciascun 
      individuo. La funzione di fitness dipende dall'algoritmo di selezione utilizzato.\\
      
      \newpage
      \begin{figure}[h!]
        \centering
        \caption{Funzione di valutazione -- pseudocodice}
        \includegraphics[scale=0.5]{Funzione di valutazione - evaluate.png}
        \label{funzioneDiValutazionePseudocodice}
      \end{figure}
   
    \newpage
      Calcola punteggio gene -- pseudocodice:
       \begin{figure}[h!]
        \centering
        \caption{Calcola punteggio gene -- pseudocodice:}
        \includegraphics[scale=0.5]{calcolaPunteggioGene.png}
        \label{calcolaPunteggioGenePseudocodice}
      \end{figure}
      
      \newpage
      
  
    \subsection{Mi fermo? -- Criteri di arresto}
    \label{MiFermo?--CriteriDiArrestoSubsection}
    Criteri di arresto utilizzati:
    \begin{enumerate}
      \item Test Ottimalità: se raggiungiamo l’ottimo allora l'algoritmo termina.
      \item Tempo di Esecuzione: dopo 4 secondi l’algoritmo termina.
    \end{enumerate}
    
  
    \subsection{Selezione}
    \label{Selezione}
    Esistono vari tipi di selezione. Andiamo ad analizzarli:
    - Roulette Wheel Selection:
    \begin{figure}[h!]
      \centering
      \caption{Roulette Wheel Selection}
      \includegraphics[scale=0.5]{Roulette Wheel Selection.png}
      \label{RouletteWheelSelection}
    \end{figure}
    \begin{enumerate}
      \item fitness$_{i}(codifica) =\frac{f_{i}(codifica)}{\sum\limits_{j = 1}^{M} {f_{j} (codifica)}}$.
      \item andiamo a girare la ruota un numero di volte uguale al numero di individui presenti nella 
            popolazione (M volte).
    \end{enumerate} 
    
    \newpage
    
    Rank Selection:\\\\
    $ fitness(codifica) = \frac {(\mid P \mid - rank(codifica) + 1)} {\sum{rank(codifica)}}$\\
    E’ applicabile anche con valutazioni negative.\\
    Costa più della Roulette Wheel Selection.\\
    
    K-way Tournament Selection:\\
    Fissato un valore $1 < K < \mid P \mid$ arbitrario, si selezionano casualmente K individui per formare un 
    cosiddetto "torneo", all’interno del quale il migliore (con valutazione massima) passerà la 
    selezione. Lo ripetiamo per $M < \mid P \mid$ tornei, quindi fino ad ottenere un totale di M individui 
    selezionati.\\\\
    
    Quale tipo di selezione utilizzeremo?\\
    Roulette Wheel Selection: poco ideale per popolazioni piccole.\\
    Rank selection: troppo lenta.\\
    Scegliamo la k-way Tournament Selection.\\
    
  
    \subsection{Crossover}
    \label{Crossover}
    Single-point Crossover:\\
    Punto di taglio  = numero casuale tra 1 e 9.\\
    Esempio: 5
    \begin{figure}[h!]
      \centering
      \caption{Single-point Crossover}
      \includegraphics[scale=0.5]{Single-point Crossover.png}
      \label{Single-pointCrossover}
    \end{figure}
    
    
    Two-point Crossover:\\
    Punti di taglio: 2 numeri casuali tra 1 e 9 diversi tra di loro.\\
    Esempio: 2, 6
    \begin{figure}[h!]
      \centering
      \caption{Two-point Crossover}
      \includegraphics[scale=0.5]{Two-point Crossover.png}
      \label{Two-pointCrossover}
    \end{figure}
    
    \newpage
    
    K-point Crossover:\\
    Punti di taglio: K numeri casuali tra 1 e 9 diversi tra loro. 1 $\leq$ K $\leq$ 9.\\
    Esempio: K = 4 $\rightarrow$ 4 numeri casuali tra 1 e 9 diversi tra di loro.\\
    Esempio: 1, 3, 5, 6
    \begin{figure}[h!]
      \centering
      \caption{K-point Crossover}
      \includegraphics[scale=0.5]{K-point Crossover.png}
      \label{K-pointCrossover}
    \end{figure}
    
    Uniform Crossover:\\
    Non si compie alcun taglio, ma il gene i-esimo di entrambi i figli sarà pari al gene i-esimo di uno 
    dei due genitori. Dato che la scelta tra i due genitori è uniforme, si avranno 2 figli 
    con circa il 50\% del materiale del primo genitore e il 50\% del secondo genitore.\\
    Esempio:
    \begin{figure}[h!]
      \centering
      \caption{Uniform Crossover}
      \includegraphics[scale=0.5]{Uniform Crossover.png}
      \label{UniformCrossover}
    \end{figure}
    
    Quale tipo di Crossover utilizzeremo?\\
    - Single-point Crossover con codifiche piccole non è l’ideale.\\
    - Two-point Crossover crea più diversità ma abbiamo sempre una codifica piccola.\\
    - K-point Crossover: Quale deve essere il valore ideale di K?\\
    Scegliamo l’Uniform Crossover. In modo che non facciamo nessun taglio 
    e c’è più diversità.\\
    
  
    \subsection{Mutazione}
    \label{Mutazione}
    Random Resetting:\\
    modifica casuale di un gene in un nuovo intero.\\
    Esempio:\\
    N = 30\\
    \begin{enumerate}
      \item Scegliamo un gene casuale da mutare: il gene in posizione 2: 14;
      \item Scegliamo uno intero casuale 0 $\leq$ x $\leq$ $\mid$vettore spazio stati$\mid$ - 1: x = 5;
      \item Mutiamo il gene 2: 14$ \rightarrow$ 5.
    \end{enumerate}
    \begin{figure}[h!]
      \centering
      \caption{Random Resetting}
      \includegraphics[scale=0.5]{Random Resetting.png}
      \label{Random Resetting}
    \end{figure}
    
    Swap:\\
    scambiamo di posto 2 geni scelti casualmente.\\
    Esempio:\\
    \begin{enumerate}
    	\item scegliamo casualmente 2 geni in posizione i e j: 
    	\begin{enumerate}
    		\item i = 1 $\rightarrow$ gene =  10;
    		\item j = 5. $\rightarrow$ gene = 4.
    	\end{enumerate}
    	\item li scambiamo di posto:
    	\begin{enumerate}
    		\item i = 5 $\rightarrow$ gene = 4;
    		\item j = 1. $\rightarrow$ gene = 10.
    	\end{enumerate}
    \end{enumerate}
    
    \begin{figure}[h!]
      \centering
      \caption{Swap}
      \includegraphics[scale=0.5]{Swap.png}
      \label{Swap}
    \end{figure}
    
    Scramble:\\
    selezioniamo una sottosequenza casuale di geni e la permutiamo casualmente.
    Esempio:\\
    \begin{enumerate}
      \item Scegliamo casualmente la sottosequenza 2 → 3 → 4 → 5.
      \item La permutiamo casualmente: 4 → 5 → 2 → 3.
    \end{enumerate}
    \begin{figure}[h!]
      \centering
      \caption{Scramble}
      \includegraphics[scale=0.5]{Scramble.png}
      \label{Scramble}
    \end{figure}
     
    Inversion:\\
    selezioniamo una sottosequenza casuale di geni e la invertiamo.\\
    Esempio:\\
    \begin{enumerate}
      \item Scegliamo casualmente la sottosequenza 2 → 3 → 4 → 5.
      \item La invertiamo: 5 → 4 → 3 → 2.
    \end{enumerate}
    \begin{figure}[h!]
      \centering
      \caption{Inversion}
      \includegraphics[scale=0.5]{Inversion.png}
      \label{Inversion}
    \end{figure}
    
    Random Mutation:\\
    modifica casuale di piu' geni in nuovi interi.\\
    Esempio:\\
    N = 30\\
    \begin{enumerate}
      \item Scegliamo casualmente i geni in posizione 0, 4, 6, 8 $ \rightarrow$ valori: 5, 0, 12, 25;
      \item Scegliamo casualmente dei nuovi valori x del valore: \\ 0 $\leq$ x $\leq$ $\mid$vettore spazio stati$\mid$ - 1 che rappresentano gli indici presenti 
      nel vettore spazio stati. Scegliamo casualmente i valori 10, 20, 5, 15.
      \item Mutiamo i geni: 0: 5 $ \rightarrow$ 10;     4: 0 $ \rightarrow$ 20;     6: 12 $ \rightarrow$ 5;     8: 25 $ \rightarrow$ 15.
    \end{enumerate}
    \begin{figure}[h!]
      \centering
      \caption{Random Mutation}
      \includegraphics[scale=0.5]{Random Mutation.png}
      \label{Random Resetting}
    \end{figure}
    
    Quale tipo di mutazione utilizzeremo?\\
    Ricordiamo che ogni gene rappresenta la posizione di uno stato nel vettore spazio stati e, uno stato contiene come parametro 
    un percorso formativo, quindi, non avrebbe senso utilizzare i seguenti tipi di mutazione:\\
    - Swap, Scramble e Inversion\\
    poiche' gli individui mutati saranno gli stessi individui di prima solo che avranno gli stessi geni scambiati di posto.\\
    Quindi, prendiamo in considerazione la seguenti mutazioni:\\
    - Random resetting;\\
    - Random mutation.\\
    Poiché con il random resetting mutiamo un unico gene, e poiché abbiamo una codifica piccola, potrebbe capitare che l'algoritmo genetico evolve poco.
    Per questo motivo ho deciso di utilizzare la seguente mutazione:\\
     - Random mutation.\\
    
     \subsection{Elitismo}
    \label{elitismoSubsection}
    Poiché gli individui d'elite non dovrebbero superare il 10\% della popolazione totale allora l'elitismo lo utilizzeremo solo se il numero della 
    popolazione totale è maggiore o uguale a 10, questo perché, il 10\% di 10 è uguale ad 1.
    Gli individui di elite non subiranno ne la selezione, ne il crossover e ne la mutazione e saranno sempre massimo 2 individui 
    (ovvero, l'individuo migliore della generazione attuale e la sua copia).
    
  
    \subsection{Preserviamo l'ammissibilita'}
    \label{PreserviamoAmmissibilitaSubsection}
    Con il crossover e la mutazione ci potrebbe essere il rischio che nelle nuove generazioni potrebbero esistere degli individui non ammissibili. 
    Nel nostro caso questo non accade perche':
    \begin{enumerate}
      \item Nel crossover: abbiamo 2 genitori che formano 2 figli contenenti i geni al 50\% del genitore 1 e al 50\% del genitore 2. 
      Poiche' ogni gene non è altro che un intero che indica la posizione di uno stato nel vettore spazio stati e uno stato ha come attributo un 
      percorso formativo esistente allora non avremo mai dei percorsi formativi inesistenti;
      \item Nella mutazione: i nuovi valori dei geni mutati saranno sempre degli indici validi poiche' rappresenteranno sempre delle posizioni degli stati nel 
      vettore spazio stati e quindi, di conseguenza, abbiamo sempre dei percorsi formativi validi.
    \end{enumerate}
    
  
    \subsection{Aggiorniamo la popolazione}
    \label{aggiorniamoPopolazioneSubsection}
    La nuova popolazione sara' costituita da M individui, ovvero:
    \begin{itemize}
      \item Sia A l'insieme degli individui di elite.
      \item Sia B l'insieme degli individui che hanno partecipato alla selezione, al crossover e alla mutazione.
      \item Nuova popolazione = A U B.
    \end{itemize}
    Quindi, aggiorniamo la vecchia popolazione con la nuova popolazione salvandoci pero' l'individuo 
    migliore della vecchia generazione se e solo se quest'ultimo e' migliore dell'ultimo individuo 
    migliore salvato (utile nel caso in cui non eseguiamo l'elitismo).\\
    N.B.: l'individuo migliore salvato non farà parte della nuova generazione.
    
  
    \subsection{Ottengo il miglior individuo}
    \label{ottengoMigliorIndividuoSubsection}
    Quando usciamo dal loop ritorniamo il miglior individuo dell'ultima generazione.\\
    nel caso in cui 0  $<$ M $<$ 10 allora confrontiamo l'individuo migliore ottenuto con l'individuo migliore salvato (non abbiamo individui d'elite).
    
  
    \subsection{Riformulazione dei passi dell'algoritmo genetico}
    \label{riformulazioneDeiPassiAlgoritmoGeneticoSubsection}
    Alla luce della progettazione fatta, possiamo riformulare il tutto con i seguenti passi 
    dell'algoritmo genetico: 
    \begin{figure}[h!]
      \centering
      \caption{Riformulazione dei passi dell'algoritmo genetico}
      \includegraphics[scale=0.5]{Riformulazione passi algoritmo genetico.png}
      \label{Riformulazione passi algoritmo genetico}
    \end{figure}
    .
   
     
\end{document}













